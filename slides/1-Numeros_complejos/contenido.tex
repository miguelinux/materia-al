% ex: ts=2 sw=2 sts=2 et filetype=tex
% SPDX-License-Identifier: CC-BY-SA-4.0

\section{Definición y origen}

\begin{frame}[c]{¿Qué son los números complejos?}

  Los \textbf{números complejos}:

  \begin{itemize}
    \item son designados con la notación $\mathbb{C}$.
    \pausa
    \item son una extensión de los números reales $\mathbb{R}$.
    \pausa
    \item forman un cuerpo algebraicamente cerrado.
  \end{itemize}
\end{frame}

\begin{frame}[c]{¿Cuál es la relación de los números complejos?}
  \[
    \text{Complejos: }\mathbb{C} \begin{cases}
      \text{Reales:} & \mathbb{R} \begin{cases}
        \text{Racionales:} & \mathbb{Q} \begin{cases}
          \text{Enteros:} & \mathbb{Z} \begin{cases}
            \text{Naturales:} & \mathbb{N} \\
            \text{Cero:} & 0 \\
            \text{Enteros negativos} & \\
            \end{cases} \\
          \text{Fraccionarios} &  \\
          \end{cases} \\
        \text{Irracionales} &  \\
        \end{cases} \\
      \text{Imaginarios} & \\
    \end{cases}
  \]
\end{frame}

\begin{frame}[c]{¿Cómo surgieron los números complejos?}

  Históricamente, los números complejos surgieron para tratar
  \underline{ecuaciones polinómicas}, tales como

  \begin{displaymath}
    x^2 + 1 = 0
  \end{displaymath}

  que \textbf{no} tienen solución real.

  \vspace{\baselineskip}
  El resultado principal que consideraremos será el
  \textbf{Teorema Fundamental del Álgebra} que asegura que toda
  ecuación polinómica con coeficientes complejos tiene, al menos,
  una solución.
\end{frame}

\begin{frame}[c]{Forma binómica}
  \begin{block}{Definición}
    Un número complejo $z \in \mathbb{C}$ es un número de la forma
    \begin{center}
      $z = a + ib$
    \end{center}
    o ($z = a + bi$), donde $ \{ a,b \in \mathbb{R}$ y
    $ i^2 = -1 \}$
  \end{block}
  \pausa
  \begin{itemize}
    \item A $i$ se le llama \emph{unidad imaginaria}
    \item Los número reales $a$ y $b$ se conocen, respectivamente, como
      \emph{parte real} y \emph{parte imaginaria}
  \end{itemize}
  \begin{displaymath}
    Re(z) = a, \qquad  Im(z) = b
  \end{displaymath}
\end{frame}

\begin{frame}[c]{Forma binómica}
  Dos números $z,w \in \mathbb{C}$, son iguales, si y sólo si,
  \begin{displaymath}
    Re(z) = Re(w), \qquad  Im(z) = Im(w)
  \end{displaymath}

  Sea $z = a + bi$.
  \begin{itemize}
    \item Si $a=0$, escribiremos $z=bi$ (\textbf{Número imaginario puro}).
    \item Si $b=0$, escribiremos $z=a + 0i$.
  \end{itemize}
\end{frame}

\begin{frame}[c]{Forma binómica}
  \begin{columns}
    \column{0.5\textwidth}
      \begin{tikzpicture}
        \coordinate (O) at (0,0);
        \draw[->] (-0.3,0) -- (4,0) coordinate[label = {below:$reales$}] ;
        \draw[->] (0,-0.3) -- (0,4) coordinate[label = {right:$imaginarios$}] ;
        \draw[densely dotted]  (1,0) node[anchor=north] {$a$} -- (1,1) ;
        \draw[densely dotted]  (0,1) node[anchor=east] {$b$} -- (1,1);
        \node[circle, fill,inner sep=1.5pt] at (1,1) {};
        \node[anchor=west] at (1,1) {$P = (a,b)$};
      \end{tikzpicture}
    \column{0.5\textwidth}
      Un número complejo $z = a+bi$ lo podemos representar por el punto $P$
      del plano que tiene por coordenadas cartesianas $(a,b)$.
  \end{columns}
\end{frame}

\begin{frame}[c]{Forma binómica}
  \begin{columns}
    \column{0.5\textwidth}
      \begin{tikzpicture}
        \coordinate (O) at (0,0);
        \draw[->] (-0.3,0) -- (4,0) coordinate[label = {below:$reales$}] ;
        \draw[->] (0,-0.3) -- (0,4) coordinate[label = {right:$imaginarios$}] ;
        \draw[-{Latex[length=3mm]}] (0,0) -- (1,1);
        \node[anchor=west] at (1,1) {$P = (a,b)$};
      \end{tikzpicture}
    \column{0.5\textwidth}
      A veces también lo representamos por el vector posición
      $\overrightarrow{OP}$ del punto $P$.

      \vspace{\baselineskip}
      Interpretado de esta manera, el plano cartesiano se le denomina
      también \textbf{plano complejo}. El eje de abscisas se suele denomina
      \textbf{eje real} y el eje de ordenadas \textbf{eje imaginario}
  \end{columns}
\end{frame}

\section{Operaciones fundamentales}

\begin{frame}[fragile]
  \frametitle{Titulo}

  \vspace{\baselineskip}
  \begin{lstlisting}[language=Python]
  \end{lstlisting}
\end{frame}

\section{Potencias de $i$, módulo o valor absoluto}

\begin{frame}[c]{Forma binómica}
  \begin{block}{Definición}
  Sea $z \in \mathbb{C}$ donde $z = a + ib = \{ a,b \in \mathbb{R}$ y $ i^2
  = -1 \}$
  \end{block}
\end{frame}

\section{Forma polar y exponencial}

\begin{frame}[c]{Forma binómica}
  \begin{block}{Definición}
  Sea $z \in \mathbb{C}$ donde $z = a + ib = \{ a,b \in \mathbb{R}$ y $ i^2
  = -1 \}$
  \end{block}
\end{frame}

\section{Teorema de De Moivre, potencias y extracción de raíces}

\begin{frame}[c]{Forma binómica}
  \begin{block}{Definición}
  Sea $z \in \mathbb{C}$ donde $z = a + ib = \{ a,b \in \mathbb{R}$ y $ i^2
  = -1 \}$
  \end{block}
\end{frame}

\section{Ecuaciones polinómicas}

\begin{frame}[c]{Forma binómica}
  \begin{block}{Definición}
  Sea $z \in \mathbb{C}$ donde $z = a + ib = \{ a,b \in \mathbb{R}$ y $ i^2
  = -1 \}$
  \end{block}
\end{frame}
