% ex: ts=2 sw=2 sts=2 et filetype=tex
% SPDX-License-Identifier: CC-BY-SA-4.0

\section{Definición}

\begin{frame}[c]{¿Qué son los números complejos?}

  Los \textbf{números complejos}:

  \begin{itemize}
    \item son designados con la notación $\mathbb{C}$.
    \pausa
    \item son una extensión de los números reales $\mathbb{R}$.
    \pausa
    \item forman un cuerpo algebraicamente cerrado.
  \end{itemize}
\end{frame}

\begin{frame}[c]{¿Cuál es la relación de los números complejos?}
  \[
    \text{Complejos: }\mathbb{C} \begin{cases}
      \text{Reales:} & \mathbb{R} \begin{cases}
        \text{Racionales:} & \mathbb{Q} \begin{cases}
          \text{Enteros:} & \mathbb{Z} \begin{cases}
            \text{Naturales:} & \mathbb{N} \\
            \text{Cero:} & 0 \\
            \text{Enteros negativos} & \\
            \end{cases} \\
          \text{Fraccionarios} &  \\
          \end{cases} \\
        \text{Irracionales} &  \\
        \end{cases} \\
      \text{Imaginarios} & \\
    \end{cases}
  \]
\end{frame}

\begin{frame}[c]{¿Cómo surgieron los números complejos?}

  Históricamente, los números complejos surgieron para tratar
  \underline{ecuaciones polinómicas}, tales como

  \begin{displaymath}
    x^2 + 1 = 0
  \end{displaymath}

  que \textbf{no} tienen solución real.

  \vspace{\baselineskip}
  El resultado principal que consideraremos será el
  \textbf{Teorema Fundamental del Álgebra} que asegura que toda
  ecuación polinómica con coeficientes complejos tiene, al menos,
  una solución.
\end{frame}

\begin{frame}[c]{Definición}
  Sea $z \in \mathbb{C}$ donde $z = a + ib = \{ a,b \in \mathbb{R}$ y $ i^2
  = -1 \}$
\end{frame}

\begin{frame}[fragile]
  \frametitle{Titulo}

  \vspace{\baselineskip}
  \begin{lstlisting}[language=Python]
  \end{lstlisting}
\end{frame}
